


% Header, overrides base

    % Make sure that the sphinx doc style knows who it inherits from.
    \def\sphinxdocclass{article}

    % Declare the document class
    \documentclass[letterpaper,10pt,english]{/opt/local/Library/Frameworks/Python.framework/Versions/3.3/lib/python3.3/site-packages/Sphinx-1.2.1-py3.3.egg/sphinx/texinputs/sphinxhowto}

    % Imports
    \usepackage[utf8]{inputenc}
    \DeclareUnicodeCharacter{00A0}{\\nobreakspace}
    \usepackage[T1]{fontenc}
    \usepackage{babel}
    \usepackage{times}
    \usepackage{import}
    \usepackage[Bjarne]{/opt/local/Library/Frameworks/Python.framework/Versions/3.3/lib/python3.3/site-packages/Sphinx-1.2.1-py3.3.egg/sphinx/texinputs/fncychap}
    \usepackage{longtable}
    \usepackage{/opt/local/Library/Frameworks/Python.framework/Versions/3.3/lib/python3.3/site-packages/Sphinx-1.2.1-py3.3.egg/sphinx/texinputs/sphinx}
    \usepackage{multirow}

    \usepackage{amsmath}
    \usepackage{amssymb}
    \usepackage{ucs}
    \usepackage{enumerate}

    % Used to make the Input/Output rules follow around the contents.
    \usepackage{needspace}

    % Pygments requirements
    \usepackage{fancyvrb}
    \usepackage{color}
    % ansi colors additions
    \definecolor{darkgreen}{rgb}{.12,.54,.11}
    \definecolor{lightgray}{gray}{.95}
    \definecolor{brown}{rgb}{0.54,0.27,0.07}
    \definecolor{purple}{rgb}{0.5,0.0,0.5}
    \definecolor{darkgray}{gray}{0.25}
    \definecolor{lightred}{rgb}{1.0,0.39,0.28}
    \definecolor{lightgreen}{rgb}{0.48,0.99,0.0}
    \definecolor{lightblue}{rgb}{0.53,0.81,0.92}
    \definecolor{lightpurple}{rgb}{0.87,0.63,0.87}
    \definecolor{lightcyan}{rgb}{0.5,1.0,0.83}

    % Needed to box output/input
    \usepackage{tikz}
        \usetikzlibrary{calc,arrows,shadows}
    \usepackage[framemethod=tikz]{mdframed}

    \usepackage{alltt}

    % Used to load and display graphics
    \usepackage{graphicx}
    \graphicspath{ {figs/} }
    \usepackage[Export]{adjustbox} % To resize

    % used so that images for notebooks which have spaces in the name can still be included
    \usepackage{grffile}


    % For formatting output while also word wrapping.
    \usepackage{listings}
    \lstset{breaklines=true}
    \lstset{basicstyle=\small\ttfamily}
    \def\smaller{\fontsize{9.5pt}{9.5pt}\selectfont}

    %Pygments definitions
    
\makeatletter
\def\PY@reset{\let\PY@it=\relax \let\PY@bf=\relax%
    \let\PY@ul=\relax \let\PY@tc=\relax%
    \let\PY@bc=\relax \let\PY@ff=\relax}
\def\PY@tok#1{\csname PY@tok@#1\endcsname}
\def\PY@toks#1+{\ifx\relax#1\empty\else%
    \PY@tok{#1}\expandafter\PY@toks\fi}
\def\PY@do#1{\PY@bc{\PY@tc{\PY@ul{%
    \PY@it{\PY@bf{\PY@ff{#1}}}}}}}
\def\PY#1#2{\PY@reset\PY@toks#1+\relax+\PY@do{#2}}

\expandafter\def\csname PY@tok@c\endcsname{\let\PY@it=\textit\def\PY@tc##1{\textcolor[rgb]{0.25,0.50,0.50}{##1}}}
\expandafter\def\csname PY@tok@sr\endcsname{\def\PY@tc##1{\textcolor[rgb]{0.73,0.40,0.53}{##1}}}
\expandafter\def\csname PY@tok@ss\endcsname{\def\PY@tc##1{\textcolor[rgb]{0.10,0.09,0.49}{##1}}}
\expandafter\def\csname PY@tok@mh\endcsname{\def\PY@tc##1{\textcolor[rgb]{0.40,0.40,0.40}{##1}}}
\expandafter\def\csname PY@tok@mi\endcsname{\def\PY@tc##1{\textcolor[rgb]{0.40,0.40,0.40}{##1}}}
\expandafter\def\csname PY@tok@sx\endcsname{\def\PY@tc##1{\textcolor[rgb]{0.00,0.50,0.00}{##1}}}
\expandafter\def\csname PY@tok@k\endcsname{\let\PY@bf=\textbf\def\PY@tc##1{\textcolor[rgb]{0.00,0.50,0.00}{##1}}}
\expandafter\def\csname PY@tok@o\endcsname{\def\PY@tc##1{\textcolor[rgb]{0.40,0.40,0.40}{##1}}}
\expandafter\def\csname PY@tok@s2\endcsname{\def\PY@tc##1{\textcolor[rgb]{0.73,0.13,0.13}{##1}}}
\expandafter\def\csname PY@tok@il\endcsname{\def\PY@tc##1{\textcolor[rgb]{0.40,0.40,0.40}{##1}}}
\expandafter\def\csname PY@tok@m\endcsname{\def\PY@tc##1{\textcolor[rgb]{0.40,0.40,0.40}{##1}}}
\expandafter\def\csname PY@tok@s\endcsname{\def\PY@tc##1{\textcolor[rgb]{0.73,0.13,0.13}{##1}}}
\expandafter\def\csname PY@tok@sb\endcsname{\def\PY@tc##1{\textcolor[rgb]{0.73,0.13,0.13}{##1}}}
\expandafter\def\csname PY@tok@sc\endcsname{\def\PY@tc##1{\textcolor[rgb]{0.73,0.13,0.13}{##1}}}
\expandafter\def\csname PY@tok@sd\endcsname{\let\PY@it=\textit\def\PY@tc##1{\textcolor[rgb]{0.73,0.13,0.13}{##1}}}
\expandafter\def\csname PY@tok@w\endcsname{\def\PY@tc##1{\textcolor[rgb]{0.73,0.73,0.73}{##1}}}
\expandafter\def\csname PY@tok@sh\endcsname{\def\PY@tc##1{\textcolor[rgb]{0.73,0.13,0.13}{##1}}}
\expandafter\def\csname PY@tok@si\endcsname{\let\PY@bf=\textbf\def\PY@tc##1{\textcolor[rgb]{0.73,0.40,0.53}{##1}}}
\expandafter\def\csname PY@tok@nn\endcsname{\let\PY@bf=\textbf\def\PY@tc##1{\textcolor[rgb]{0.00,0.00,1.00}{##1}}}
\expandafter\def\csname PY@tok@vc\endcsname{\def\PY@tc##1{\textcolor[rgb]{0.10,0.09,0.49}{##1}}}
\expandafter\def\csname PY@tok@go\endcsname{\def\PY@tc##1{\textcolor[rgb]{0.53,0.53,0.53}{##1}}}
\expandafter\def\csname PY@tok@vg\endcsname{\def\PY@tc##1{\textcolor[rgb]{0.10,0.09,0.49}{##1}}}
\expandafter\def\csname PY@tok@gi\endcsname{\def\PY@tc##1{\textcolor[rgb]{0.00,0.63,0.00}{##1}}}
\expandafter\def\csname PY@tok@gd\endcsname{\def\PY@tc##1{\textcolor[rgb]{0.63,0.00,0.00}{##1}}}
\expandafter\def\csname PY@tok@ge\endcsname{\let\PY@it=\textit}
\expandafter\def\csname PY@tok@vi\endcsname{\def\PY@tc##1{\textcolor[rgb]{0.10,0.09,0.49}{##1}}}
\expandafter\def\csname PY@tok@kt\endcsname{\def\PY@tc##1{\textcolor[rgb]{0.69,0.00,0.25}{##1}}}
\expandafter\def\csname PY@tok@cm\endcsname{\let\PY@it=\textit\def\PY@tc##1{\textcolor[rgb]{0.25,0.50,0.50}{##1}}}
\expandafter\def\csname PY@tok@cp\endcsname{\def\PY@tc##1{\textcolor[rgb]{0.74,0.48,0.00}{##1}}}
\expandafter\def\csname PY@tok@mf\endcsname{\def\PY@tc##1{\textcolor[rgb]{0.40,0.40,0.40}{##1}}}
\expandafter\def\csname PY@tok@cs\endcsname{\let\PY@it=\textit\def\PY@tc##1{\textcolor[rgb]{0.25,0.50,0.50}{##1}}}
\expandafter\def\csname PY@tok@kn\endcsname{\let\PY@bf=\textbf\def\PY@tc##1{\textcolor[rgb]{0.00,0.50,0.00}{##1}}}
\expandafter\def\csname PY@tok@gt\endcsname{\def\PY@tc##1{\textcolor[rgb]{0.00,0.27,0.87}{##1}}}
\expandafter\def\csname PY@tok@gu\endcsname{\let\PY@bf=\textbf\def\PY@tc##1{\textcolor[rgb]{0.50,0.00,0.50}{##1}}}
\expandafter\def\csname PY@tok@kc\endcsname{\let\PY@bf=\textbf\def\PY@tc##1{\textcolor[rgb]{0.00,0.50,0.00}{##1}}}
\expandafter\def\csname PY@tok@kd\endcsname{\let\PY@bf=\textbf\def\PY@tc##1{\textcolor[rgb]{0.00,0.50,0.00}{##1}}}
\expandafter\def\csname PY@tok@gr\endcsname{\def\PY@tc##1{\textcolor[rgb]{1.00,0.00,0.00}{##1}}}
\expandafter\def\csname PY@tok@gs\endcsname{\let\PY@bf=\textbf}
\expandafter\def\csname PY@tok@s1\endcsname{\def\PY@tc##1{\textcolor[rgb]{0.73,0.13,0.13}{##1}}}
\expandafter\def\csname PY@tok@ni\endcsname{\let\PY@bf=\textbf\def\PY@tc##1{\textcolor[rgb]{0.60,0.60,0.60}{##1}}}
\expandafter\def\csname PY@tok@no\endcsname{\def\PY@tc##1{\textcolor[rgb]{0.53,0.00,0.00}{##1}}}
\expandafter\def\csname PY@tok@err\endcsname{\def\PY@bc##1{\setlength{\fboxsep}{0pt}\fcolorbox[rgb]{1.00,0.00,0.00}{1,1,1}{\strut ##1}}}
\expandafter\def\csname PY@tok@nl\endcsname{\def\PY@tc##1{\textcolor[rgb]{0.63,0.63,0.00}{##1}}}
\expandafter\def\csname PY@tok@nc\endcsname{\let\PY@bf=\textbf\def\PY@tc##1{\textcolor[rgb]{0.00,0.00,1.00}{##1}}}
\expandafter\def\csname PY@tok@nb\endcsname{\def\PY@tc##1{\textcolor[rgb]{0.00,0.50,0.00}{##1}}}
\expandafter\def\csname PY@tok@na\endcsname{\def\PY@tc##1{\textcolor[rgb]{0.49,0.56,0.16}{##1}}}
\expandafter\def\csname PY@tok@nf\endcsname{\def\PY@tc##1{\textcolor[rgb]{0.00,0.00,1.00}{##1}}}
\expandafter\def\csname PY@tok@ne\endcsname{\let\PY@bf=\textbf\def\PY@tc##1{\textcolor[rgb]{0.82,0.25,0.23}{##1}}}
\expandafter\def\csname PY@tok@nd\endcsname{\def\PY@tc##1{\textcolor[rgb]{0.67,0.13,1.00}{##1}}}
\expandafter\def\csname PY@tok@gh\endcsname{\let\PY@bf=\textbf\def\PY@tc##1{\textcolor[rgb]{0.00,0.00,0.50}{##1}}}
\expandafter\def\csname PY@tok@nv\endcsname{\def\PY@tc##1{\textcolor[rgb]{0.10,0.09,0.49}{##1}}}
\expandafter\def\csname PY@tok@nt\endcsname{\let\PY@bf=\textbf\def\PY@tc##1{\textcolor[rgb]{0.00,0.50,0.00}{##1}}}
\expandafter\def\csname PY@tok@kp\endcsname{\def\PY@tc##1{\textcolor[rgb]{0.00,0.50,0.00}{##1}}}
\expandafter\def\csname PY@tok@gp\endcsname{\let\PY@bf=\textbf\def\PY@tc##1{\textcolor[rgb]{0.00,0.00,0.50}{##1}}}
\expandafter\def\csname PY@tok@c1\endcsname{\let\PY@it=\textit\def\PY@tc##1{\textcolor[rgb]{0.25,0.50,0.50}{##1}}}
\expandafter\def\csname PY@tok@ow\endcsname{\let\PY@bf=\textbf\def\PY@tc##1{\textcolor[rgb]{0.67,0.13,1.00}{##1}}}
\expandafter\def\csname PY@tok@bp\endcsname{\def\PY@tc##1{\textcolor[rgb]{0.00,0.50,0.00}{##1}}}
\expandafter\def\csname PY@tok@mo\endcsname{\def\PY@tc##1{\textcolor[rgb]{0.40,0.40,0.40}{##1}}}
\expandafter\def\csname PY@tok@kr\endcsname{\let\PY@bf=\textbf\def\PY@tc##1{\textcolor[rgb]{0.00,0.50,0.00}{##1}}}
\expandafter\def\csname PY@tok@se\endcsname{\let\PY@bf=\textbf\def\PY@tc##1{\textcolor[rgb]{0.73,0.40,0.13}{##1}}}

\def\PYZbs{\char`\\}
\def\PYZus{\char`\_}
\def\PYZob{\char`\{}
\def\PYZcb{\char`\}}
\def\PYZca{\char`\^}
\def\PYZam{\char`\&}
\def\PYZlt{\char`\<}
\def\PYZgt{\char`\>}
\def\PYZsh{\char`\#}
\def\PYZpc{\char`\%}
\def\PYZdl{\char`\$}
\def\PYZhy{\char`\-}
\def\PYZsq{\char`\'}
\def\PYZdq{\char`\"}
\def\PYZti{\char`\~}
% for compatibility with earlier versions
\def\PYZat{@}
\def\PYZlb{[}
\def\PYZrb{]}
\makeatother


    %Set pygments styles if needed...
    
        \definecolor{nbframe-border}{rgb}{0.867,0.867,0.867}
        \definecolor{nbframe-bg}{rgb}{0.969,0.969,0.969}
        \definecolor{nbframe-in-prompt}{rgb}{0.0,0.0,0.502}
        \definecolor{nbframe-out-prompt}{rgb}{0.545,0.0,0.0}

        \newenvironment{ColorVerbatim}
        {\begin{mdframed}[%
            roundcorner=1.0pt, %
            backgroundcolor=nbframe-bg, %
            userdefinedwidth=1\linewidth, %
            leftmargin=0.1\linewidth, %
            innerleftmargin=0pt, %
            innerrightmargin=0pt, %
            linecolor=nbframe-border, %
            linewidth=1pt, %
            usetwoside=false, %
            everyline=true, %
            innerlinewidth=3pt, %
            innerlinecolor=nbframe-bg, %
            middlelinewidth=1pt, %
            middlelinecolor=nbframe-bg, %
            outerlinewidth=0.5pt, %
            outerlinecolor=nbframe-border, %
            needspace=0pt
        ]}
        {\end{mdframed}}
        
        \newenvironment{InvisibleVerbatim}
        {\begin{mdframed}[leftmargin=0.1\linewidth,innerleftmargin=3pt,innerrightmargin=3pt, userdefinedwidth=1\linewidth, linewidth=0pt, linecolor=white, usetwoside=false]}
        {\end{mdframed}}

        \renewenvironment{Verbatim}[1][\unskip]
        {\begin{alltt}\smaller}
        {\end{alltt}}
    

    % Help prevent overflowing lines due to urls and other hard-to-break 
    % entities.  This doesn't catch everything...
    \sloppy

    % Document level variables
    \title{cmax}
    \date{April 11, 2014}
    \release{}
    \author{Jelle Aalbers and Chris Tunnell}
    \renewcommand{\releasename}{}

    % TODO: Add option for the user to specify a logo for his/her export.
    \newcommand{\sphinxlogo}{}

    % Make the index page of the document.
    \makeindex

    % Import sphinx document type specifics.
     


% Body

    % Start of the document
    \begin{document}

        
            \maketitle
        

        


        
        \part{$\bar{C}_\text{max}(1 - \alpha, \mu)$}Computing $\bar{C}_\text{max}(C, \mu)$ for optimum interval calculation,
where $\mu$ is the number of expected events and $1 - \alpha$ is how
frequently you reject the null hypothesis when it is true.The single-event energy spectrum, that is, the probability density
function which tells us which energy depositions are likely to occur, is
independent of the chosen WIMP model -- we always expect a simple
exponential recoil spectrum.

The number of dark matter events detected does depend on the WIMP mass
and cross-section. We know, however, that it must follow a Poisson
distribution, which leaves the Poisson mean (which equals the expected
number of events) as the only parameter left to estimate. From an upper
limit on this mean, an upper limit curve in the dark matter mass --
cross-section plane can be computed.

\begin{itemize}
\itemsep1pt\parskip0pt\parsep0pt
\item
  A list\_of\_energies list of reconstructed energy depositions of
  single events (from here on simply `energies'), either measured during
  some run of an actual detector, or generated using Monte Carlo.)
\item
  An interval is an interval in energy space.
\item
  The size of an interval is the fraction of energies expected in that
  interval. Clearly, this depends on which energy spectrum we assume,
  but is independent of the Poisson mean we are trying to constrain. By
  definition this is a number between 0 and 1.
\item
  The K-largest interval of a run is the largest interval containing K
  events in that run. Recall our definition of size: a `large' interval
  is one which is unusually empty in that run. Clearly k-largest
  intervals will terminate at (or technically, just before) an observed
  energy, or at one of the boundaries of our energy space. Again, which
  interval in a run is the k--largest, depends on our energy spectrum,
  but not on our Poisson mean.
\item
  The extremeness of a K-largest interval is the probability of finding
  the K-largest interval in a run to be smaller. This clearly does
  depend on the Poisson mean: if we expect very few events, large gap
  sizes are more likely. Clearly extremeness is a number between 0 and
  1; values close to 1 indicate unusually large intervals, that is,
  usually large (almost-)empty regions in the measured energies. For
  example, if the extremeness of a k-largest interval in a run is 0.8,
  that means that 80\% of runs have k-largest intervals which are
  smaller than the k-largest interval in this run.
\item
  The optimum interval statistic of a run is extremity of the most
  extreme k-largest interval in a run.
\item
  The extremeness of the optimum interval statistic is the probability
  of finding a lower optimum interval statistic, that is, of finding the
  optimum interval in a run to be less extreme.
\end{itemize}

The max gap method rejects a theory (places a mean outside the upper
limit) based on a run if the 0-largest interval (the largest gap) is too
extreme.

The optimum interval method rejects a theory based on a run if the
optimum interval statistic is too large.

\begin{itemize}
\itemsep1pt\parskip0pt\parsep0pt
\item
  The energy cumulant $\epsilon(E)$ is the fraction of energies expected
  below the energy $E$. Whatever the (1-normalized) energy distribution
  $dN/dE$, $dN/d\epsilon$ is uniform{[}0,1{]}, where $0$ and $1$
  correspond to the boundaries of our experimental range.
\end{itemize}

    % Make sure that atleast 4 lines are below the HR
    \needspace{4\baselineskip}

    
        \vspace{6pt}
        \makebox[0.1\linewidth]{\smaller\hfill\tt\color{nbframe-in-prompt}In\hspace{4pt}{[}16{]}:\hspace{4pt}}\\*
        \vspace{-2.65\baselineskip}
        \begin{ColorVerbatim}
            \vspace{-0.7\baselineskip}
            \begin{Verbatim}[commandchars=\\\{\}]
\PY{k+kn}{import} \PY{n+nn}{functools}
\PY{k+kn}{from} \PY{n+nn}{scipy.optimize} \PY{k+kn}{import} \PY{n}{brenth}
\PY{k+kn}{import} \PY{n+nn}{matplotlib.pyplot} \PY{k+kn}{as} \PY{n+nn}{plt}
\PY{k+kn}{import} \PY{n+nn}{numpy} \PY{k+kn}{as} \PY{n+nn}{np}
\PY{k+kn}{import} \PY{n+nn}{pickle}
\end{Verbatim}

            
                \vspace{-0.2\baselineskip}
            
        \end{ColorVerbatim}
    


    % Make sure that atleast 4 lines are below the HR
    \needspace{4\baselineskip}

    
        \vspace{6pt}
        \makebox[0.1\linewidth]{\smaller\hfill\tt\color{nbframe-in-prompt}In\hspace{4pt}{[}17{]}:\hspace{4pt}}\\*
        \vspace{-2.65\baselineskip}
        \begin{ColorVerbatim}
            \vspace{-0.7\baselineskip}
            \begin{Verbatim}[commandchars=\\\{\}]
\PY{k}{def} \PY{n+nf}{kLargestIntervals}\PY{p}{(}\PY{n}{list\PYZus{}of\PYZus{}energies}\PY{p}{,} \PY{n}{spectrumCDF} \PY{o}{=} \PY{k}{lambda} \PY{n}{x}\PY{p}{:} \PY{n}{x}\PY{p}{)}\PY{p}{:}
    \PY{l+s+sd}{\PYZdq{}\PYZdq{}\PYZdq{}}
\PY{l+s+sd}{    Returns a list of the sizes of the K\PYZhy{}largest intervals in that run according to the energy spectrum (given as a CDF).}
\PY{l+s+sd}{    That is, kLargestIntervals(...)[i] is the size of the largest interval containing i events, where ‘largest’ is defined above.}
\PY{l+s+sd}{    }
\PY{l+s+sd}{    * Transform energies to energy cumulants}
\PY{l+s+sd}{    * Add events at 0 and 1}
\PY{l+s+sd}{    * Foreach k, compute interval sizes, take max}
\PY{l+s+sd}{    \PYZdq{}\PYZdq{}\PYZdq{}}
    \PY{n}{answer} \PY{o}{=} \PY{p}{\PYZob{}}\PY{p}{\PYZcb{}}
    
    \PY{n}{list\PYZus{}of\PYZus{}energies}\PY{o}{.}\PY{n}{sort}\PY{p}{(}\PY{p}{)}
    
    \PY{n}{energy\PYZus{}cumulants} \PY{o}{=} \PY{n}{spectrumCDF}\PY{p}{(}\PY{n}{list\PYZus{}of\PYZus{}energies}\PY{p}{)}
        
    \PY{k}{for} \PY{n}{interval\PYZus{}size} \PY{o+ow}{in} \PY{n+nb}{range}\PY{p}{(}\PY{n+nb}{len}\PY{p}{(}\PY{n}{energy\PYZus{}cumulants}\PY{p}{)}\PY{p}{)}\PY{p}{:}
        \PY{k}{if} \PY{p}{(}\PY{l+m+mi}{1} \PY{o}{+} \PY{n}{interval\PYZus{}size}\PY{p}{)} \PY{o}{\PYZgt{}}\PY{o}{=} \PY{n+nb}{len}\PY{p}{(}\PY{n}{energy\PYZus{}cumulants}\PY{p}{)}\PY{p}{:}
            \PY{k}{continue}
            
        \PY{n}{temp\PYZus{}data} \PY{o}{=} \PY{n}{energy\PYZus{}cumulants}\PY{o}{.}\PY{n}{copy}\PY{p}{(}\PY{p}{)}
        \PY{n}{gap\PYZus{}sizes} \PY{o}{=} \PY{n}{temp\PYZus{}data}\PY{p}{[}\PY{p}{(}\PY{l+m+mi}{1}\PY{o}{+}\PY{n}{interval\PYZus{}size}\PY{p}{)}\PY{p}{:}\PY{p}{]} \PY{o}{\PYZhy{}} \PY{n}{temp\PYZus{}data}\PY{p}{[}\PY{l+m+mi}{0}\PY{p}{:}\PY{o}{\PYZhy{}}\PY{l+m+mi}{1}\PY{o}{*}\PY{p}{(}\PY{l+m+mi}{1} \PY{o}{+} \PY{n}{interval\PYZus{}size}\PY{p}{)}\PY{p}{]} 

        \PY{n}{answer}\PY{p}{[}\PY{n}{interval\PYZus{}size}\PY{p}{]} \PY{o}{=} \PY{n}{np}\PY{o}{.}\PY{n}{max}\PY{p}{(}\PY{n}{gap\PYZus{}sizes}\PY{p}{)}

    \PY{k}{return} \PY{n}{answer}

\PY{k}{assert} \PY{n}{kLargestIntervals}\PY{p}{(}\PY{n}{np}\PY{o}{.}\PY{n}{array}\PY{p}{(}\PY{p}{[}\PY{l+m+mf}{0.0}\PY{p}{,} \PY{l+m+mf}{0.1}\PY{p}{,} \PY{l+m+mf}{0.2}\PY{p}{,} \PY{l+m+mf}{0.84}\PY{p}{,} \PY{l+m+mf}{0.85}\PY{p}{]}\PY{p}{)}\PY{p}{)}\PY{p}{[}\PY{l+m+mi}{0}\PY{p}{]} \PY{o}{==} \PY{p}{(}\PY{l+m+mf}{0.84} \PY{o}{\PYZhy{}} \PY{l+m+mf}{0.2}\PY{p}{)}  \PY{c}{\PYZsh{} test 0}
\PY{k}{assert} \PY{n}{kLargestIntervals}\PY{p}{(}\PY{n}{np}\PY{o}{.}\PY{n}{array}\PY{p}{(}\PY{p}{[}\PY{l+m+mf}{0.0}\PY{p}{,} \PY{l+m+mf}{0.1}\PY{p}{,} \PY{l+m+mf}{0.2}\PY{p}{,} \PY{l+m+mf}{0.84}\PY{p}{,} \PY{l+m+mf}{0.85}\PY{p}{]}\PY{p}{)}\PY{p}{)}\PY{p}{[}\PY{l+m+mi}{2}\PY{p}{]} \PY{o}{==} \PY{p}{(}\PY{l+m+mf}{0.84} \PY{o}{\PYZhy{}} \PY{l+m+mf}{0.0}\PY{p}{)}  \PY{c}{\PYZsh{} test 2}
\PY{k}{assert} \PY{n}{kLargestIntervals}\PY{p}{(}\PY{n}{np}\PY{o}{.}\PY{n}{array}\PY{p}{(}\PY{p}{[}\PY{l+m+mf}{0.85}\PY{p}{,} \PY{l+m+mf}{0.0}\PY{p}{,} \PY{l+m+mf}{0.1}\PY{p}{,} \PY{l+m+mf}{0.84}\PY{p}{,} \PY{l+m+mf}{0.2}\PY{p}{]}\PY{p}{)}\PY{p}{)}\PY{p}{[}\PY{l+m+mi}{2}\PY{p}{]} \PY{o}{==} \PY{p}{(}\PY{l+m+mf}{0.84} \PY{o}{\PYZhy{}} \PY{l+m+mf}{0.0}\PY{p}{)}  \PY{c}{\PYZsh{} test unsorted}
\end{Verbatim}

            
                \vspace{-0.2\baselineskip}
            
        \end{ColorVerbatim}
    


    % Make sure that atleast 4 lines are below the HR
    \needspace{4\baselineskip}

    
        \vspace{6pt}
        \makebox[0.1\linewidth]{\smaller\hfill\tt\color{nbframe-in-prompt}In\hspace{4pt}{[}18{]}:\hspace{4pt}}\\*
        \vspace{-2.65\baselineskip}
        \begin{ColorVerbatim}
            \vspace{-0.7\baselineskip}
            \begin{Verbatim}[commandchars=\\\{\}]
\PY{k}{def} \PY{n+nf}{extremenessOfInterval}\PY{p}{(}\PY{n}{x}\PY{p}{,} \PY{n}{k}\PY{p}{,} \PY{n}{mu}\PY{p}{)}\PY{p}{:}
    \PY{l+s+sd}{\PYZdq{}\PYZdq{}\PYZdq{}}
\PY{l+s+sd}{    Returns the extremeness of a k\PYZhy{}largest interval of size, if the poisson mean is mu.}
\PY{l+s+sd}{    }
\PY{l+s+sd}{    (Number of itvSizes[mu][k] smaller than size) / mcTrials[mu]}
\PY{l+s+sd}{    }
\PY{l+s+sd}{    x \PYZhy{} also size in above comment}
\PY{l+s+sd}{    k \PYZhy{} gap (rename k)}
\PY{l+s+sd}{    \PYZdq{}\PYZdq{}\PYZdq{}}
    \PY{c}{\PYZsh{} [0] is because where returns list, where [0] is answer}
    \PY{k}{if} \PY{n}{k} \PY{o+ow}{not} \PY{o+ow}{in} \PY{n}{itvSizes}\PY{p}{[}\PY{n}{mu}\PY{p}{]}\PY{p}{:}
        \PY{k}{return} \PY{l+m+mi}{0}
    
    \PY{k}{return} \PY{n}{np}\PY{o}{.}\PY{n}{where}\PY{p}{(}\PY{n}{itvSizes}\PY{p}{[}\PY{n}{mu}\PY{p}{]}\PY{p}{[}\PY{n}{k}\PY{p}{]} \PY{o}{\PYZlt{}} \PY{n}{x}\PY{p}{)}\PY{p}{[}\PY{l+m+mi}{0}\PY{p}{]}\PY{o}{.}\PY{n}{size} \PY{o}{/} \PY{n}{mcTrials}\PY{p}{[}\PY{n}{mu}\PY{p}{]}
\end{Verbatim}

            
                \vspace{-0.2\baselineskip}
            
        \end{ColorVerbatim}
    


    % Make sure that atleast 4 lines are below the HR
    \needspace{4\baselineskip}

    
        \vspace{6pt}
        \makebox[0.1\linewidth]{\smaller\hfill\tt\color{nbframe-in-prompt}In\hspace{4pt}{[}19{]}:\hspace{4pt}}\\*
        \vspace{-2.65\baselineskip}
        \begin{ColorVerbatim}
            \vspace{-0.7\baselineskip}
            \begin{Verbatim}[commandchars=\\\{\}]
\PY{k}{def} \PY{n+nf}{optimumItvStatistic}\PY{p}{(}\PY{n}{list\PYZus{}of\PYZus{}energies}\PY{p}{,} \PY{n}{mu}\PY{p}{,} \PY{n}{spectrumCDF} \PY{o}{=} \PY{k}{lambda} \PY{n}{x}\PY{p}{:} \PY{n}{x}\PY{p}{)}\PY{p}{:}
    \PY{l+s+sd}{\PYZdq{}\PYZdq{}\PYZdq{}}
\PY{l+s+sd}{    Returns the optimum interval statistic of the run.}
\PY{l+s+sd}{    }
\PY{l+s+sd}{    Max of extremenssOfInterval\PYZsq{}s}
\PY{l+s+sd}{    \PYZdq{}\PYZdq{}\PYZdq{}}
    \PY{k}{return} \PY{n}{np}\PY{o}{.}\PY{n}{max}\PY{p}{(}\PY{p}{[}\PY{n}{extremenessOfInterval}\PY{p}{(}\PY{n}{x}\PY{p}{,} \PY{n}{k}\PY{p}{,} \PY{n}{mu}\PY{p}{)} \PY{k}{for} \PY{n}{k}\PY{p}{,} \PY{n}{x} \PY{o+ow}{in} \PY{n}{kLargestIntervals}\PY{p}{(}\PY{n}{list\PYZus{}of\PYZus{}energies}\PY{p}{,} \PY{n}{spectrumCDF}\PY{p}{)}\PY{o}{.}\PY{n}{items}\PY{p}{(}\PY{p}{)}\PY{p}{]}\PY{p}{)}
\end{Verbatim}

            
                \vspace{-0.2\baselineskip}
            
        \end{ColorVerbatim}
    


    % Make sure that atleast 4 lines are below the HR
    \needspace{4\baselineskip}

    
        \vspace{6pt}
        \makebox[0.1\linewidth]{\smaller\hfill\tt\color{nbframe-in-prompt}In\hspace{4pt}{[}20{]}:\hspace{4pt}}\\*
        \vspace{-2.65\baselineskip}
        \begin{ColorVerbatim}
            \vspace{-0.7\baselineskip}
            \begin{Verbatim}[commandchars=\\\{\}]
\PY{k}{def} \PY{n+nf}{extremenessOfOptItvStat}\PY{p}{(}\PY{n}{stat}\PY{p}{,} \PY{n}{mu}\PY{p}{)}\PY{p}{:}
    \PY{l+s+sd}{\PYZdq{}\PYZdq{}\PYZdq{}}
\PY{l+s+sd}{    Returns the extremeness of the optimum interval statistic stat, given mu}
\PY{l+s+sd}{    }
\PY{l+s+sd}{    (Number of optItvs[mu] smaller than stat) / mcTrials[mu]}
\PY{l+s+sd}{    \PYZdq{}\PYZdq{}\PYZdq{}}
    \PY{k}{return} \PY{n}{np}\PY{o}{.}\PY{n}{where}\PY{p}{(}\PY{n}{optItvs}\PY{p}{[}\PY{n}{mu}\PY{p}{]} \PY{o}{\PYZlt{}} \PY{n}{stat}\PY{p}{)}\PY{p}{[}\PY{l+m+mi}{0}\PY{p}{]}\PY{o}{.}\PY{n}{size} \PY{o}{/} \PY{n}{mcTrials}\PY{p}{[}\PY{n}{mu}\PY{p}{]}
\end{Verbatim}

            
                \vspace{-0.2\baselineskip}
            
        \end{ColorVerbatim}
    


    % Make sure that atleast 4 lines are below the HR
    \needspace{4\baselineskip}

    
        \vspace{6pt}
        \makebox[0.1\linewidth]{\smaller\hfill\tt\color{nbframe-in-prompt}In\hspace{4pt}{[}26{]}:\hspace{4pt}}\\*
        \vspace{-2.65\baselineskip}
        \begin{ColorVerbatim}
            \vspace{-0.7\baselineskip}
            \begin{Verbatim}[commandchars=\\\{\}]
\PY{k}{def} \PY{n+nf}{optItvUpperLimit}\PY{p}{(}\PY{n}{list\PYZus{}of\PYZus{}energies}\PY{p}{,} \PY{n}{c}\PY{p}{,} \PY{n}{spectrumCDF} \PY{o}{=} \PY{k}{lambda} \PY{n}{x}\PY{p}{:} \PY{n}{x}\PY{p}{,}
                     \PY{n}{n} \PY{o}{=} \PY{l+m+mi}{1000}\PY{p}{)}\PY{p}{:}
    \PY{l+s+sd}{\PYZdq{}\PYZdq{}\PYZdq{}}
\PY{l+s+sd}{    Returns the c\PYZhy{} confidence upper limit on mu using optimum interval}
\PY{l+s+sd}{    }
\PY{l+s+sd}{    For which mu is extremenessOfOptItvStat( optimumItvStatistic(run), mu ) = c}
\PY{l+s+sd}{    }
\PY{l+s+sd}{    c \PYZhy{} e.g., 0.9}
\PY{l+s+sd}{    \PYZdq{}\PYZdq{}\PYZdq{}}
    \PY{k}{def} \PY{n+nf}{f}\PY{p}{(}\PY{n}{mu}\PY{p}{,} \PY{n}{list\PYZus{}of\PYZus{}energies}\PY{p}{,} \PY{n}{c}\PY{p}{,} \PY{n}{spectrumCDF}\PY{p}{,} \PY{n}{n}\PY{p}{)}\PY{p}{:}
        \PY{n}{generate\PYZus{}table}\PY{p}{(}\PY{n}{mu}\PY{p}{,} \PY{n}{n}\PY{p}{)}
        \PY{n}{x} \PY{o}{=} \PY{n}{optimumItvStatistic}\PY{p}{(}\PY{n}{list\PYZus{}of\PYZus{}energies}\PY{p}{,} \PY{n}{mu}\PY{p}{,} \PY{n}{spectrumCDF}\PY{p}{)}
        \PY{n}{prob} \PY{o}{=} \PY{n}{extremenessOfOptItvStat}\PY{p}{(}\PY{n}{x}\PY{p}{,} \PY{n}{mu}\PY{p}{)}
        \PY{k}{return} \PY{n}{prob} \PY{o}{\PYZhy{}} \PY{n}{c}

    \PY{n}{mu} \PY{o}{=} \PY{l+m+mi}{0}

    \PY{k}{for} \PY{n}{mu} \PY{o+ow}{in} \PY{n}{np}\PY{o}{.}\PY{n}{arange}\PY{p}{(}\PY{l+m+mi}{10}\PY{p}{,} \PY{l+m+mi}{2} \PY{o}{*} \PY{n}{list\PYZus{}of\PYZus{}energies}\PY{o}{.}\PY{n}{size}\PY{p}{)}\PY{p}{:}
        \PY{k}{if} \PY{n}{f}\PY{p}{(}\PY{n}{mu}\PY{p}{,} \PY{n}{list\PYZus{}of\PYZus{}energies}\PY{p}{,} \PY{n}{c}\PY{p}{,} \PY{n}{spectrumCDF}\PY{p}{,} \PY{n}{n}\PY{p}{)} \PY{o}{\PYZgt{}} \PY{l+m+mi}{0}\PY{p}{:}
            \PY{k}{print}\PY{p}{(}\PY{l+s}{\PYZsq{}}\PY{l+s}{Found seed mu=}\PY{l+s+si}{\PYZpc{}f}\PY{l+s}{\PYZsq{}} \PY{o}{\PYZpc{}} \PY{n}{mu}\PY{p}{)}
            \PY{k}{break}
        
    \PY{k}{try}\PY{p}{:}
        \PY{n}{xsec} \PY{o}{=} \PY{n}{brenth}\PY{p}{(}\PY{n}{f}\PY{p}{,} \PY{n}{mu} \PY{o}{\PYZhy{}} \PY{l+m+mi}{5}\PY{p}{,} \PY{n}{mu} \PY{o}{+} \PY{l+m+mi}{5}\PY{p}{,}
                  \PY{n}{args}\PY{o}{=}\PY{p}{(}\PY{n}{list\PYZus{}of\PYZus{}energies}\PY{p}{,} \PY{n}{c}\PY{p}{,} \PY{n}{spectrumCDF}\PY{p}{,} \PY{n}{n}\PY{p}{)}\PY{p}{,}
                  \PY{n}{xtol}\PY{o}{=}\PY{l+m+mf}{1e\PYZhy{}2}\PY{p}{)} 
        \PY{k}{print}\PY{p}{(}\PY{l+s}{\PYZsq{}}\PY{l+s}{Improved xsec:}\PY{l+s}{\PYZsq{}}\PY{p}{,} \PY{n}{xsec}\PY{p}{)}
    \PY{k}{except}\PY{p}{:}
        \PY{k}{print}\PY{p}{(}\PY{l+s}{\PYZdq{}}\PY{l+s}{ERROR: could not minimize}\PY{l+s}{\PYZdq{}}\PY{p}{,} \PY{n}{mu}\PY{p}{)}
        \PY{k}{return} \PY{n}{mu}
    \PY{k}{return} \PY{n}{xsec}



    
\end{Verbatim}

            
                \vspace{-0.2\baselineskip}
            
        \end{ColorVerbatim}
    


    % Make sure that atleast 4 lines are below the HR
    \needspace{4\baselineskip}

    
        \vspace{6pt}
        \makebox[0.1\linewidth]{\smaller\hfill\tt\color{nbframe-in-prompt}In\hspace{4pt}{[}22{]}:\hspace{4pt}}\\*
        \vspace{-2.65\baselineskip}
        \begin{ColorVerbatim}
            \vspace{-0.7\baselineskip}
            \begin{Verbatim}[commandchars=\\\{\}]
\PY{k}{def} \PY{n+nf}{generate\PYZus{}trial\PYZus{}experiment}\PY{p}{(}\PY{n}{mu}\PY{p}{,} \PY{n}{n}\PY{p}{)}\PY{p}{:}
    \PY{n}{trials} \PY{o}{=} \PY{p}{[}\PY{p}{]}

    \PY{k}{for} \PY{n}{index} \PY{o+ow}{in} \PY{n+nb}{range}\PY{p}{(}\PY{n}{n}\PY{p}{)}\PY{p}{:}
        \PY{n}{this\PYZus{}mu} \PY{o}{=} \PY{n}{np}\PY{o}{.}\PY{n}{random}\PY{o}{.}\PY{n}{poisson}\PY{p}{(}\PY{n}{mu}\PY{p}{)}
        
        \PY{n}{rand\PYZus{}numbers} \PY{o}{=} \PY{n}{np}\PY{o}{.}\PY{n}{random}\PY{o}{.}\PY{n}{random}\PY{p}{(}\PY{n}{size}\PY{o}{=}\PY{n}{this\PYZus{}mu}\PY{p}{)}
        \PY{n}{rand\PYZus{}numbers} \PY{o}{=} \PY{n}{np}\PY{o}{.}\PY{n}{append}\PY{p}{(}\PY{n}{rand\PYZus{}numbers}\PY{p}{,} \PY{p}{[}\PY{l+m+mf}{0.0}\PY{p}{,} \PY{l+m+mf}{1.0}\PY{p}{]}\PY{p}{)}
        \PY{n}{rand\PYZus{}numbers}\PY{o}{.}\PY{n}{sort}\PY{p}{(}\PY{p}{)}
        \PY{n}{trials}\PY{o}{.}\PY{n}{append}\PY{p}{(}\PY{n}{rand\PYZus{}numbers}\PY{p}{)}

    \PY{k}{return} \PY{n}{trials}
\end{Verbatim}

            
                \vspace{-0.2\baselineskip}
            
        \end{ColorVerbatim}
    
\section{Monte Carlo for populating itvSizes{[}$\mu${]}{[}$k${]} and
optItvs{[}$\mu${]}}

    % Make sure that atleast 4 lines are below the HR
    \needspace{4\baselineskip}

    
        \vspace{6pt}
        \makebox[0.1\linewidth]{\smaller\hfill\tt\color{nbframe-in-prompt}In\hspace{4pt}{[}36{]}:\hspace{4pt}}\\*
        \vspace{-2.65\baselineskip}
        \begin{ColorVerbatim}
            \vspace{-0.7\baselineskip}
            \begin{Verbatim}[commandchars=\\\{\}]
\PY{k}{def} \PY{n+nf}{get\PYZus{}filename}\PY{p}{(}\PY{p}{)}\PY{p}{:}
    \PY{k}{return} \PY{l+s}{\PYZsq{}}\PY{l+s}{saved\PYZus{}intervals.p}\PY{l+s}{\PYZsq{}}

\PY{k}{def} \PY{n+nf}{load\PYZus{}table\PYZus{}from\PYZus{}disk}\PY{p}{(}\PY{p}{)}\PY{p}{:}
    \PY{k}{global} \PY{n}{itvSizes}
    \PY{k}{global} \PY{n}{optItvs}
    \PY{k}{global} \PY{n}{mcTrials}
    
    \PY{k}{if} \PY{n}{os}\PY{o}{.}\PY{n}{path}\PY{o}{.}\PY{n}{exists}\PY{p}{(}\PY{n}{get\PYZus{}filename}\PY{p}{(}\PY{p}{)}\PY{p}{)}\PY{p}{:}
        \PY{n}{f} \PY{o}{=} \PY{n+nb}{open}\PY{p}{(}\PY{n}{get\PYZus{}filename}\PY{p}{(}\PY{p}{)}\PY{p}{,} \PY{l+s}{\PYZsq{}}\PY{l+s}{rb}\PY{l+s}{\PYZsq{}}\PY{p}{)}
        \PY{n}{itvSizes} \PY{o}{=} \PY{n}{pickle}\PY{o}{.}\PY{n}{load}\PY{p}{(}\PY{n}{f}\PY{p}{)}
        \PY{n}{optItvs} \PY{o}{=} \PY{n}{pickle}\PY{o}{.}\PY{n}{load}\PY{p}{(}\PY{n}{f}\PY{p}{)}
        \PY{n}{mcTrials} \PY{o}{=} \PY{n}{pickle}\PY{o}{.}\PY{n}{load}\PY{p}{(}\PY{n}{f}\PY{p}{)}
        \PY{n}{f}\PY{o}{.}\PY{n}{close}\PY{p}{(}\PY{p}{)}

    
\PY{k}{def} \PY{n+nf}{write\PYZus{}table\PYZus{}to\PYZus{}disk}\PY{p}{(}\PY{p}{)}\PY{p}{:}
    \PY{n}{f} \PY{o}{=} \PY{n+nb}{open}\PY{p}{(}\PY{n}{get\PYZus{}filename}\PY{p}{(}\PY{p}{)}\PY{p}{,} \PY{l+s}{\PYZsq{}}\PY{l+s}{wb}\PY{l+s}{\PYZsq{}}\PY{p}{)}
    \PY{n}{pickle}\PY{o}{.}\PY{n}{dump}\PY{p}{(}\PY{n}{itvSizes}\PY{p}{,} \PY{n}{f}\PY{p}{)}
    \PY{n}{pickle}\PY{o}{.}\PY{n}{dump}\PY{p}{(}\PY{n}{optItvs}\PY{p}{,} \PY{n}{f}\PY{p}{)}
    \PY{n}{pickle}\PY{o}{.}\PY{n}{dump}\PY{p}{(}\PY{n}{mcTrials}\PY{p}{,} \PY{n}{f}\PY{p}{)}
    \PY{n}{f}\PY{o}{.}\PY{n}{close}\PY{p}{(}\PY{p}{)}

\PY{n}{itvSizes} \PY{o}{=} \PY{p}{\PYZob{}}\PY{p}{\PYZcb{}}
\PY{n}{optItvs} \PY{o}{=} \PY{p}{\PYZob{}}\PY{p}{\PYZcb{}}
\PY{n}{mcTrials} \PY{o}{=} \PY{p}{\PYZob{}}\PY{p}{\PYZcb{}}
\PY{n}{load\PYZus{}table\PYZus{}from\PYZus{}disk}\PY{p}{(}\PY{p}{)}    
    
\PY{k}{def} \PY{n+nf}{generate\PYZus{}table}\PY{p}{(}\PY{n}{mu}\PY{p}{,} \PY{n}{n}\PY{p}{)}\PY{p}{:}
    \PY{l+s+sd}{\PYZdq{}\PYZdq{}\PYZdq{}    \PYZsh{}Generate trial runs\PYZdq{}\PYZdq{}\PYZdq{}}    
    \PY{k}{if} \PY{n}{mu} \PY{o+ow}{in} \PY{n}{mcTrials} \PY{o+ow}{and} \PY{n}{mcTrials}\PY{p}{[}\PY{n}{mu}\PY{p}{]} \PY{o}{\PYZgt{}}\PY{o}{=} \PY{n}{n}\PY{p}{:}
        \PY{k}{return}

    \PY{k}{print}\PY{p}{(}\PY{l+s}{\PYZdq{}}\PY{l+s}{Generating}\PY{l+s}{\PYZdq{}}\PY{p}{,} \PY{n}{mu}\PY{p}{)}

    \PY{n}{mcTrials}\PY{p}{[}\PY{n}{mu}\PY{p}{]} \PY{o}{=} \PY{n}{n}
    \PY{n}{trials} \PY{o}{=} \PY{n}{generate\PYZus{}trial\PYZus{}experiment}\PY{p}{(}\PY{n}{mu}\PY{p}{,} \PY{n}{mcTrials}\PY{p}{[}\PY{n}{mu}\PY{p}{]}\PY{p}{)}

    \PY{n}{itvSizes}\PY{p}{[}\PY{n}{mu}\PY{p}{]} \PY{o}{=} \PY{p}{\PYZob{}}\PY{p}{\PYZcb{}}
    \PY{n}{optItvs}\PY{p}{[}\PY{n}{mu}\PY{p}{]} \PY{o}{=} \PY{p}{[}\PY{p}{]}

    \PY{k}{for} \PY{n}{trial} \PY{o+ow}{in} \PY{n}{trials}\PY{p}{:}
        \PY{n}{intermediate\PYZus{}result} \PY{o}{=} \PY{n}{kLargestIntervals}\PY{p}{(}\PY{n}{trial}\PY{p}{)}
        
        \PY{k}{for} \PY{n}{k}\PY{p}{,} \PY{n}{v} \PY{o+ow}{in} \PY{n}{intermediate\PYZus{}result}\PY{o}{.}\PY{n}{items}\PY{p}{(}\PY{p}{)}\PY{p}{:}
            \PY{k}{if} \PY{n}{k} \PY{o+ow}{not} \PY{o+ow}{in} \PY{n}{itvSizes}\PY{p}{[}\PY{n}{mu}\PY{p}{]}\PY{p}{:}
                \PY{n}{itvSizes}\PY{p}{[}\PY{n}{mu}\PY{p}{]}\PY{p}{[}\PY{n}{k}\PY{p}{]} \PY{o}{=} \PY{p}{[}\PY{p}{]}

            \PY{n}{itvSizes}\PY{p}{[}\PY{n}{mu}\PY{p}{]}\PY{p}{[}\PY{n}{k}\PY{p}{]}\PY{o}{.}\PY{n}{append}\PY{p}{(}\PY{n}{v}\PY{p}{)}
    
    \PY{c}{\PYZsh{} Numpy\PYZhy{}ize it}
    \PY{k}{for} \PY{n}{k}\PY{p}{,} \PY{n}{array} \PY{o+ow}{in} \PY{n}{itvSizes}\PY{p}{[}\PY{n}{mu}\PY{p}{]}\PY{o}{.}\PY{n}{items}\PY{p}{(}\PY{p}{)}\PY{p}{:}
        \PY{n}{itvSizes}\PY{p}{[}\PY{n}{mu}\PY{p}{]}\PY{p}{[}\PY{n}{k}\PY{p}{]} \PY{o}{=} \PY{n}{np}\PY{o}{.}\PY{n}{array}\PY{p}{(}\PY{n}{array}\PY{p}{)}
    
    \PY{k}{for} \PY{n}{trial} \PY{o+ow}{in} \PY{n}{trials}\PY{p}{:}
        \PY{n}{optItvs}\PY{p}{[}\PY{n}{mu}\PY{p}{]}\PY{o}{.}\PY{n}{append}\PY{p}{(}\PY{n}{optimumItvStatistic}\PY{p}{(}\PY{n}{trial}\PY{p}{,} \PY{n}{mu}\PY{p}{)}\PY{p}{)}
        
    \PY{c}{\PYZsh{} Numpy\PYZhy{}ize it}
    \PY{n}{optItvs}\PY{p}{[}\PY{n}{mu}\PY{p}{]} \PY{o}{=} \PY{n}{np}\PY{o}{.}\PY{n}{array}\PY{p}{(}\PY{n}{optItvs}\PY{p}{[}\PY{n}{mu}\PY{p}{]}\PY{p}{)}
    
    


\PY{k}{def} \PY{n+nf}{cache\PYZus{}values}\PY{p}{(}\PY{n}{my\PYZus{}max}\PY{o}{=}\PY{l+m+mi}{200}\PY{p}{,} \PY{n}{n}\PY{o}{=}\PY{l+m+mi}{100}\PY{p}{)}\PY{p}{:}
    \PY{k}{for} \PY{n}{i} \PY{o+ow}{in} \PY{n+nb}{range}\PY{p}{(}\PY{l+m+mi}{3}\PY{p}{,} \PY{n}{my\PYZus{}max}\PY{p}{)}\PY{p}{:}
        \PY{n}{generate\PYZus{}table}\PY{p}{(}\PY{n}{i}\PY{p}{,} \PY{n}{n}\PY{p}{)}
    \PY{n}{write\PYZus{}table\PYZus{}to\PYZus{}disk}\PY{p}{(}\PY{p}{)}
\end{Verbatim}

            
                \vspace{-0.2\baselineskip}
            
        \end{ColorVerbatim}
    


    % Make sure that atleast 4 lines are below the HR
    \needspace{4\baselineskip}

    
        \vspace{6pt}
        \makebox[0.1\linewidth]{\smaller\hfill\tt\color{nbframe-in-prompt}In\hspace{4pt}{[}52{]}:\hspace{4pt}}\\*
        \vspace{-2.65\baselineskip}
        \begin{ColorVerbatim}
            \vspace{-0.7\baselineskip}
            \begin{Verbatim}[commandchars=\\\{\}]
\PY{k}{def} \PY{n+nf}{plot\PYZus{}something}\PY{p}{(}\PY{p}{)}\PY{p}{:}
    \PY{n}{x}\PY{p}{,} \PY{n}{y} \PY{o}{=} \PY{p}{[}\PY{p}{]}\PY{p}{,} \PY{p}{[}\PY{p}{]}
    
    \PY{k}{for} \PY{n}{mu} \PY{o+ow}{in} \PY{n}{np}\PY{o}{.}\PY{n}{linspace}\PY{p}{(}\PY{l+m+mf}{3.5}\PY{p}{,} \PY{l+m+mf}{6.5}\PY{p}{,} \PY{l+m+mi}{20}\PY{p}{)}\PY{p}{:}
        \PY{n}{generate\PYZus{}table}\PY{p}{(}\PY{n}{mu}\PY{p}{,} \PY{l+m+mi}{10000}\PY{p}{)}
        \PY{n}{x}\PY{o}{.}\PY{n}{append}\PY{p}{(}\PY{n}{mu}\PY{p}{)}
        
        \PY{n}{a} \PY{o}{=} \PY{n}{brenth}\PY{p}{(}\PY{k}{lambda} \PY{n}{x}\PY{p}{:} \PY{n}{extremenessOfOptItvStat}\PY{p}{(}\PY{n}{x}\PY{p}{,} \PY{n}{mu}\PY{p}{)} \PY{o}{\PYZhy{}} \PY{l+m+mf}{0.9}\PY{p}{,}
                   \PY{l+m+mi}{0}\PY{p}{,}
                   \PY{l+m+mi}{1}\PY{p}{,}
                   \PY{n}{xtol}\PY{o}{=}\PY{l+m+mf}{1e\PYZhy{}2}\PY{p}{)}
        
        \PY{n}{y}\PY{o}{.}\PY{n}{append}\PY{p}{(}\PY{n}{a}\PY{p}{)}
        
    \PY{n}{plt}\PY{o}{.}\PY{n}{plot}\PY{p}{(}\PY{n}{x}\PY{p}{,}\PY{n}{y}\PY{p}{)}
    \PY{n}{plt}\PY{o}{.}\PY{n}{xscale}\PY{p}{(}\PY{l+s}{\PYZsq{}}\PY{l+s}{log}\PY{l+s}{\PYZsq{}}\PY{p}{)}
    \PY{n}{plt}\PY{o}{.}\PY{n}{xlim}\PY{p}{(}\PY{l+m+mf}{3.5}\PY{p}{,} \PY{l+m+mf}{6.5}\PY{p}{)}
    
\PY{c}{\PYZsh{}plot\PYZus{}something()}
\end{Verbatim}

            
                \vspace{-0.2\baselineskip}
            
        \end{ColorVerbatim}
    

    

        % If the first block is an image, minipage the image.  Else
        % request a certain amount of space for the input text.
        \needspace{4\baselineskip}
        
        

            % Add document contents.
            
                \begin{InvisibleVerbatim}
                \vspace{-0.5\baselineskip}
    \begin{center}
    \includegraphics[max size={\textwidth}{\textheight}]{cmax_files/cmax_12_0.png}
    \par
    \end{center}
    
            \end{InvisibleVerbatim}
            
        
    


    % Make sure that atleast 4 lines are below the HR
    \needspace{4\baselineskip}

    
        \vspace{6pt}
        \makebox[0.1\linewidth]{\smaller\hfill\tt\color{nbframe-in-prompt}In\hspace{4pt}{[}71{]}:\hspace{4pt}}\\*
        \vspace{-2.65\baselineskip}
        \begin{ColorVerbatim}
            \vspace{-0.7\baselineskip}
            \begin{Verbatim}[commandchars=\\\{\}]
\PY{k}{def} \PY{n+nf}{simple\PYZus{}test\PYZus{}uniform}\PY{p}{(}\PY{p}{)}\PY{p}{:}
    \PY{n}{test\PYZus{}list\PYZus{}of\PYZus{}energies} \PY{o}{=} \PY{n}{generate\PYZus{}trial\PYZus{}experiment}\PY{p}{(}\PY{n}{mu}\PY{o}{=}\PY{l+m+mi}{100}\PY{p}{,} \PY{n}{n}\PY{o}{=}\PY{l+m+mi}{1}\PY{p}{)}\PY{p}{[}\PY{l+m+mi}{0}\PY{p}{]}
    \PY{k}{print}\PY{p}{(}\PY{n+nb}{len}\PY{p}{(}\PY{n}{test\PYZus{}list\PYZus{}of\PYZus{}energies}\PY{p}{)}\PY{p}{)}
    \PY{n}{answer} \PY{o}{=} \PY{n}{optItvUpperLimit}\PY{p}{(}\PY{n}{test\PYZus{}list\PYZus{}of\PYZus{}energies}\PY{p}{,} \PY{l+m+mf}{0.9}\PY{p}{)}
\end{Verbatim}

            
                \vspace{-0.2\baselineskip}
            
        \end{ColorVerbatim}
    

    

        % If the first block is an image, minipage the image.  Else
        % request a certain amount of space for the input text.
        \needspace{4\baselineskip}
        
        

            % Add document contents.
            
                \begin{InvisibleVerbatim}
                \vspace{-0.5\baselineskip}
\begin{alltt}107
testing mu=3.000000
\ldots 0.000000, so continuing\ldots
testing mu=4.000000
\ldots 0.000000, so continuing\ldots
testing mu=5.000000
\ldots 0.000000, so continuing\ldots
testing mu=6.000000
\ldots 0.000000, so continuing\ldots
testing mu=7.000000
\ldots 0.000000, so continuing\ldots
testing mu=8.000000
\ldots 0.000000, so continuing\ldots
testing mu=9.000000
\ldots 0.000000, so continuing\ldots
testing mu=10.000000
\ldots 0.000000, so continuing\ldots
testing mu=11.000000
\ldots 0.000000, so continuing\ldots
testing mu=12.000000
\ldots 0.000000, so continuing\ldots
testing mu=13.000000
\ldots 0.000000, so continuing\ldots
testing mu=14.000000
\ldots 0.000000, so continuing\ldots
testing mu=15.000000
\ldots 0.000000, so continuing\ldots
testing mu=16.000000
\ldots 0.000000, so continuing\ldots
testing mu=17.000000
\ldots 0.000000, so continuing\ldots
testing mu=18.000000
\ldots 0.000000, so continuing\ldots
testing mu=19.000000
\ldots 0.000000, so continuing\ldots
testing mu=20.000000
\ldots 0.000000, so continuing\ldots
testing mu=21.000000
\ldots 0.000000, so continuing\ldots
testing mu=22.000000
\ldots 0.000000, so continuing\ldots
testing mu=23.000000
Generating 23
\ldots 0.000000, so continuing\ldots
testing mu=24.000000
Generating 24
\ldots 0.000000, so continuing\ldots
testing mu=25.000000
Generating 25
\ldots 0.000000, so continuing\ldots
testing mu=26.000000
Generating 26
\ldots 0.000000, so continuing\ldots
testing mu=27.000000
Generating 27
\ldots 0.000000, so continuing\ldots
testing mu=28.000000
Generating 28
\ldots 0.000000, so continuing\ldots
testing mu=29.000000
Generating 29
\ldots 0.000000, so continuing\ldots
testing mu=30.000000
Generating 30
\ldots 0.000000, so continuing\ldots
testing mu=31.000000
Generating 31
\ldots 0.000000, so continuing\ldots
testing mu=32.000000
Generating 32
\ldots 0.000000, so continuing\ldots
testing mu=33.000000
Generating 33
\ldots 0.000000, so continuing\ldots
testing mu=34.000000
Generating 34
\ldots 0.000000, so continuing\ldots
testing mu=35.000000
Generating 35
\ldots 0.000000, so continuing\ldots
testing mu=36.000000
Generating 36
\ldots 0.000000, so continuing\ldots
testing mu=37.000000
Generating 37
\ldots 0.000000, so continuing\ldots
testing mu=38.000000
Generating 38
\ldots 0.000000, so continuing\ldots
testing mu=39.000000
Generating 39
\ldots 0.000000, so continuing\ldots
testing mu=40.000000
Generating 40
\ldots 0.000000, so continuing\ldots
testing mu=41.000000
Generating 41
\ldots 0.000000, so continuing\ldots
testing mu=42.000000
Generating 42
\ldots 0.000000, so continuing\ldots
testing mu=43.000000
Generating 43
\ldots 0.000000, so continuing\ldots
testing mu=44.000000
Generating 44
\ldots 0.000000, so continuing\ldots
testing mu=45.000000
Generating 45
\ldots 0.000000, so continuing\ldots
testing mu=46.000000
Generating 46
\ldots 0.000000, so continuing\ldots
testing mu=47.000000
Generating 47
\ldots 0.000000, so continuing\ldots
testing mu=48.000000
Generating 48
\ldots 0.000000, so continuing\ldots
testing mu=49.000000
Generating 49
\ldots 0.000000, so continuing\ldots
testing mu=50.000000
Generating 50
\ldots 0.000000, so continuing\ldots
testing mu=51.000000
Generating 51
\ldots 0.001000, so continuing\ldots
testing mu=52.000000
Generating 52
\ldots 0.000000, so continuing\ldots
testing mu=53.000000
Generating 53
\ldots 0.000000, so continuing\ldots
testing mu=54.000000
Generating 54
\ldots 0.002000, so continuing\ldots
testing mu=55.000000
Generating 55
\ldots 0.001000, so continuing\ldots
testing mu=56.000000
Generating 56
\ldots 0.002000, so continuing\ldots
testing mu=57.000000
Generating 57
\ldots 0.003000, so continuing\ldots
testing mu=58.000000
Generating 58
\ldots 0.000000, so continuing\ldots
testing mu=59.000000
Generating 59
\ldots 0.005000, so continuing\ldots
testing mu=60.000000
Generating 60
\ldots 0.002000, so continuing\ldots
testing mu=61.000000
Generating 61
\ldots 0.007000, so continuing\ldots
testing mu=62.000000
Generating 62
\ldots 0.006000, so continuing\ldots
testing mu=63.000000
Generating 63
\ldots 0.009000, so continuing\ldots
testing mu=64.000000
Generating 64
\ldots 0.013000, so continuing\ldots
testing mu=65.000000
Generating 65
\ldots 0.012000, so continuing\ldots
testing mu=66.000000
Generating 66
\ldots 0.031000, so continuing\ldots
testing mu=67.000000
Generating 67
\ldots 0.036000, so continuing\ldots
testing mu=68.000000
Generating 68
\ldots 0.034000, so continuing\ldots
testing mu=69.000000
Generating 69
\ldots 0.038000, so continuing\ldots
testing mu=70.000000
Generating 70
\ldots 0.043000, so continuing\ldots
testing mu=71.000000
Generating 71
\ldots 0.054000, so continuing\ldots
testing mu=72.000000
Generating 72
\ldots 0.059000, so continuing\ldots
testing mu=73.000000
Generating 73
\ldots 0.052000, so continuing\ldots
testing mu=74.000000
Generating 74
\ldots 0.072000, so continuing\ldots
testing mu=75.000000
Generating 75
\ldots 0.094000, so continuing\ldots
testing mu=76.000000
Generating 76
\ldots 0.101000, so continuing\ldots
testing mu=77.000000
Generating 77
\ldots 0.121000, so continuing\ldots
testing mu=78.000000
Generating 78
\ldots 0.144000, so continuing\ldots
testing mu=79.000000
Generating 79
\ldots 0.159000, so continuing\ldots
testing mu=80.000000
Generating 80
\ldots 0.177000, so continuing\ldots
testing mu=81.000000
Generating 81
\ldots 0.192000, so continuing\ldots
testing mu=82.000000
Generating 82
\ldots 0.221000, so continuing\ldots
testing mu=83.000000
Generating 83
\ldots 0.232000, so continuing\ldots
testing mu=84.000000
Generating 84
\ldots 0.268000, so continuing\ldots
testing mu=85.000000
Generating 85
\ldots 0.257000, so continuing\ldots
testing mu=86.000000
Generating 86
\ldots 0.270000, so continuing\ldots
testing mu=87.000000
Generating 87
\ldots 0.310000, so continuing\ldots
testing mu=88.000000
Generating 88
\ldots 0.343000, so continuing\ldots
testing mu=89.000000
Generating 89
\ldots 0.391000, so continuing\ldots
testing mu=90.000000
Generating 90
\ldots 0.455000, so continuing\ldots
testing mu=91.000000
Generating 91
\ldots 0.418000, so continuing\ldots
testing mu=92.000000
Generating 92
\ldots 0.467000, so continuing\ldots
testing mu=93.000000
Generating 93
\ldots 0.492000, so continuing\ldots
testing mu=94.000000
Generating 94
\ldots 0.523000, so continuing\ldots
testing mu=95.000000
Generating 95
\ldots 0.528000, so continuing\ldots
testing mu=96.000000
Generating 96
\ldots 0.536000, so continuing\ldots
testing mu=97.000000
Generating 97
\ldots 0.572000, so continuing\ldots
testing mu=98.000000
Generating 98
\ldots 0.576000, so continuing\ldots
testing mu=99.000000
Generating 99
\ldots 0.644000, so continuing\ldots
testing mu=100.000000
Generating 100
\ldots 0.644000, so continuing\ldots
testing mu=101.000000
Generating 101
\ldots 0.704000, so continuing\ldots
testing mu=102.000000
Generating 102
\ldots 0.663000, so continuing\ldots
testing mu=103.000000
Generating 103
\ldots 0.698000, so continuing\ldots
testing mu=104.000000
Generating 104
\ldots 0.708000, so continuing\ldots
testing mu=105.000000
Generating 105
\ldots 0.726000, so continuing\ldots
testing mu=106.000000
Generating 106
\ldots 0.748000, so continuing\ldots
testing mu=107.000000
Generating 107
\ldots 0.772000, so continuing\ldots
testing mu=108.000000
Generating 108
\ldots 0.802000, so continuing\ldots
testing mu=109.000000
Generating 109
\ldots 0.806000, so continuing\ldots
testing mu=110.000000
Generating 110
\ldots 0.832000, so continuing\ldots
testing mu=111.000000
Generating 111
\ldots 0.844000, so continuing\ldots
testing mu=112.000000
Generating 112
\ldots 0.832000, so continuing\ldots
testing mu=113.000000
Generating 113
\ldots 0.854000, so continuing\ldots
testing mu=114.000000
Generating 114
\ldots 0.877000, so continuing\ldots
testing mu=115.000000
Generating 115
\ldots 0.838000, so continuing\ldots
testing mu=116.000000
Generating 116
\ldots 0.899000, so continuing\ldots
testing mu=117.000000
Generating 117
\ldots 0.882000, so continuing\ldots
testing mu=118.000000
Generating 118
Found 118.000000 -> 0.902000
\end{alltt}

            \end{InvisibleVerbatim}
            
        
    

        

        \renewcommand{\indexname}{Index}
        \printindex

    % End of document
    \end{document}


